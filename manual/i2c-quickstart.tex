\documentclass[twoside,12pt,letterpaper]{refart}

\usepackage[
  paper=letterpaper,
  top=14mm, bottom=16mm,
  inner=16mm, outer=16mm,
  headheight=0pt, headsep=0pt, includehead=false
]{geometry}

\usepackage{titling}
\setlength{\droptitle}{-3.5em}

\usepackage{titlesec}
\titleformat{\section}[block]{\large\bfseries}{}{0pt}{}
\titleformat{\subsection}[block]{\normalsize\bfseries}{}{0pt}{}

\setlength{\parskip}{0.35\baselineskip}
\setlength{\parindent}{0pt}
\usepackage{enumitem}
\setlist{nosep,itemsep=0.2ex,topsep=0.4ex,partopsep=0ex}

\usepackage[final]{microtype}
\linespread{0.97}

\addtolength{\textheight}{\baselineskip}

\usepackage{hyperref}
\usepackage{tikz}
\usetikzlibrary{shapes.geometric,arrows,positioning,calc}
\tikzset{
  qa/.cd,
  width/.store in=\qaW,          width=0.42\textwidth,   % overall card width
  pad/.store in=\qaPad,          pad=1.1em,              % inner padding
  heightlines/.store in=\qaHL,   heightlines=9,          % interior height in baselines
  check/.style = {draw, inner sep=0pt,
                  minimum width=.95em, minimum height=.95em},
}

\usepackage{relsize}
\usepackage{ifthen}
\usepackage{makeidx}
\usepackage{bytefield}

\usepackage{graphicx}
\graphicspath{ {./} }
\usepackage{multirow}

\usepackage{booktabs,array}
\newcolumntype{C}[1]{>{\centering\arraybackslash}p{#1}}

\usepackage{fontspec}
\setmainfont{EB Garamond}
\setmonofont{Courier Prime}[
  Scale=MatchLowercase,
  LetterSpace=2.0
]

\newfontfamily\sixtyfour{Sixtyfour}[
  Scale=MatchUppercase
]
\DeclareTextFontCommand{\codefont}{\sixtyfour}

\ExplSyntaxOn
\cs_new:Npn \m88 {
    {\sixtyfour M88}
}
\ExplSyntaxOff
\newcommand{\itwoc}{I\textsuperscript{2}C}

\makeatletter
\renewcommand{\maketitle}{%
  \begingroup\parindent=0pt
    \vspace*{0pt}%
    \begin{center}
      {\huge\bfseries \@title \par}%
    \end{center}
  \endgroup
}
\makeatother

\pagestyle{empty}

\newcommand{\QACardMM}[1]{%
\begin{tikzpicture}[remember picture, overlay, font=\fontsize{10}{12}\selectfont]
  % Sizes (mm) — tweak these
  \def\CardW{70mm}
  \def\CardH{50mm}
  \def\Pad{5mm}
  \def\RowGap{6mm}
  \def\BoxSize{4mm}

  % Outer frame, anchored at bottom-right of the page
  \node[anchor=south east, draw, rounded corners=1mm,
        minimum width=\CardW, minimum height=\CardH,
        inner sep=\Pad, #1] (qa) at (current page.south east) {};

  % Title + rule
  \node[anchor=north, yshift=-0.5*\Pad, font=\bfseries] (title) at (qa.north) {QA Tested};
  \draw
    ($([xshift=\Pad]qa.north west |- title.south)$)
      --
    ($([xshift=-\Pad]qa.north east |- title.south)$);

  % Checklist origin
  \coordinate (list) at ($ (qa.north west) + (\Pad,-12mm) $);

  % Precompute half box for vertical centering
  \pgfmathsetlengthmacro{\halfbox}{0.5*\BoxSize}

  % Rows (avoid \label; compute offsets as lengths)
  \foreach \txt [count=\i] in
    { {Power on OK},
      {I/O base set to \texttt{3A0}},
      {\itwoc{} TX/RX OK},
      {STEMMA + DE-9 OK},
      {7-segment demo succeeds} }{%
      \pgfmathsetlengthmacro{\yoff}{(\i-1)*\RowGap}
      % checkbox (top-left at list+(0,-yoff))
      \draw ($(list)+(0,-\yoff)$) rectangle ++(\BoxSize,-\BoxSize);
      % label: centered to checkbox vertically
      \node[anchor=west]
        at ($(list)+(\BoxSize+2mm,-\yoff-\halfbox)$) {\txt};
  }

  % Footer
  \node[anchor=south west] at ($ (qa.south west) + (\Pad,2.5mm) $)
    {\small Date:\ \rule{18mm}{0.4pt}\quad Initials:\ \rule{10mm}{0.4pt}};
\end{tikzpicture}%
}

\title{{ \Huge \m88 } \\ ISA-to-\itwoc{} Bridge Quickstart}
\author{}
\date{}

\begin{document}
\maketitle
\raggedright

\begin{center}
Thank you for purchasing an \m88 ISA-to-\itwoc{} card! \\ You're a few short steps away from interfacing
your IBM PC/AT or compatible with an external \itwoc{} bus.
\end{center}

\section*{Initial Configuration}

The ISA-to-\itwoc{} bridge card includes a DIP switch to configure the card's base I/O port address.
Ports in the range \texttt{000} to \texttt{3FC} can be configured. Your card has been preset to port
\texttt{3A0}, but you may need to adjust this for your system configuration to avoid conflicts.

\begin{center}
    \begin{tabular}{ c r|c|c|c|c|c|c|c|c }
        \multirow{2}{*}{\larger[1] Port \texttt{3A0}}
        & \textbf{Switch} & A9 & A8 & A7 & A6 & A5 & A4 & A3 & A2 \\
        \cline{2-10}
        & \textbf{State} & ON & ON & ON & OFF & ON & OFF & OFF & OFF
    \end{tabular}
\end{center}

\section*{Connecting the Card}

Insert the card to any available ISA slot on your computer's motherboard and securely fasten the card's
bracket as needed for your computer case. Two options are provided for connectivity to the \itwoc{} bus:
a DE-9 connector and a 4-pin ``STEMMA'' connector. With the computer powered off, connect your \itwoc{}
devices to the appropriate connector on the card.

\begin{center}
\begin{tabular}{@{}l C{16mm} C{22mm}@{}}
\toprule
\textbf{Signal} & \textbf{\raisebox{-0.35\height}{\includegraphics[height=6mm]{stemma}}} & \textbf{DE-9 pin(s)} \\
\midrule
+5VDC & 1 & 7 \\
GND & 2 & 1, 2, 3, 4 \\
SDA & 3 & 6 \\
SCL & 4 & 8 \\
NC  & -- & 5 \\
\bottomrule
\end{tabular}

\vspace{0.4ex}
\footnotesize
STEMMA pin 1 is closest to the ISA edge connector. +5VDC is sourced from the ISA bus.
\end{center}

\section*{Programming the Card}

Obtain supporting software from GitHub at
https://github.com/miselin/m88, including:

\begin{itemize}
    \item \texttt{i2c.exe}, an MS-DOS command-line utility to interact with \itwoc{} devices.
    \item \texttt{i2c.c}, an example C program demonstrating how to interact with the card.
\end{itemize}

The NXP PCF8584 datasheet is highly recommended reading as it offers a detailed programming
guide.

\section*{Resources}

\begin{itemize}
    \item The NXP PCF8584 datasheet for detailed technical specifications
    \item The \m88 website, https://m88.computer
    \item Reach out on Tindie at https://www.tindie.com/stores/m88
    \item Email me directly at \verb|matthew@theiselins.net|
\end{itemize}

\QACardMM{xshift=-8mm,yshift=8mm}

\end{document}
