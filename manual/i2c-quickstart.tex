\documentclass[twoside,10pt,letterpaper]{refart}
\usepackage{ifthen}
\usepackage[colorlinks=true,citecolor=blue,linkcolor=blue]{hyperref}
\usepackage{bytefield}
\usepackage{makeidx}
\usepackage[T1]{fontenc}
\usepackage[lining]{ebgaramond}
\usepackage{tikz}
\usetikzlibrary{shapes.geometric, arrows}
\usepackage{relsize}

\usepackage{fontspec}
\setmonofont{Courier Prime}

\setlength{\columnseprule}{0.4pt}

\newfontfamily\sixtyfour{Sixtyfour}
\DeclareTextFontCommand{\codefont}{\sixtyfour}

\newcommand{\forceindent}[1][3em]{\leavevmode{\parindent=#1\indent}}

\ExplSyntaxOn
\cs_new:Npn \m88 {
    {\sixtyfour \smaller[1] M88}
}
\ExplSyntaxOff
\newcommand{\itwoc}{I\textsuperscript{2}C}

\makeatletter
\def\@maketitle{
  \newpage
  \null
  \begin{center}
  \let \footnote \thanks
    {\LARGE \bfseries {\@title} \par}
  \end{center}
  \par
}
\makeatother

\tikzstyle{startstop} = [rectangle, rounded corners, minimum width=3cm, minimum height=1cm,text centered, text width=3cm, draw=black]
\tikzstyle{io} = [trapezium, trapezium left angle=70, trapezium right angle=110, minimum width=3cm, minimum height=1cm, text centered, text width=3cm, draw=black]
\tikzstyle{process} = [rectangle, minimum width=3cm, minimum height=1cm, text centered, text width=3cm, draw=black]
\tikzstyle{decision} = [diamond, minimum width=3cm, minimum height=1cm, text centered, text width=3cm, aspect=3, draw=black]
\tikzstyle{arrow} = [thick,->,>=stealth]

\title{\m88 ISA-to-\itwoc{} Bridge Card Quickstart}
\author{}
\date{}

\fullpage

\begin{document}
\maketitle
\raggedright

Thank you for either purchasing or building an \m88 ISA-to-\itwoc{} card! You're a few short steps away
from interfacing your IBM PC/AT or compatible with an external \itwoc{} bus. The card is built around
the NXP PCF8584 \itwoc{} controller. If you wish to write your own software to interact with the card,
you may refer to the PCF8584 datasheet for more information.

\section*{Initial Configuration}

The ISA-to-\itwoc{} bridge card includes a DIP switch to configure the card's base I/O port address.
Ports in the range \texttt{000} to \texttt{3FF} can be configured. Your configuration needs
may vary depending on other peripherals installed on the system, but generally port \texttt{2A0}
works.

The DIP switch positions for port \texttt{0x3A0} are as follows:

\begin{center}
    \begin{tabular}{ c|c|c|c|c|c|c|c|c }
        \textbf{DIP Switch} & A9 & A8 & A7 & A6 & A5 & A4 & A3 & A2 \\
        \hline
        \textbf{Position} & 1 & 1 & 1 & 0 & 1 & 0 & 0 & 0
    \end{tabular}
\end{center}

\section*{Connecting the Card}

Insert the card to any available ISA slot in your IBM PC/AT or compatible. Securely fasten the card's
bracket using a screw or other fastener, depending on the slot and your computer's case design.

Two options are provided for connectivity to the \itwoc{} bus: a DE-9 connector and a 4-pin "STEMMA" connector.
With the computer powered off, connect your \itwoc{} devices to the appropriate connector on the card.

\subsection*{Pinouts}

todo

\section*{Software Installation}

You can obtain supporting software at the GitHub repository for the \m88 project, https://github.com/miselin/m88.

The package includes:

\begin{enumerate}
    \item \texttt{i2c.exe} - a command-line utility to interact with \itwoc{} devices.
    \item \texttt{i2c.lib} - a library exposing the card's functionality for use in your own programs.
    \item \texttt{i2c.h} - C header file for use with \texttt{i2c.lib}.
\end{enumerate}

\section*{Using \texttt{i2c.exe}}

The \texttt{i2c.exe} utility allows you to get started with interfacing your \itwoc{} devices quickly.

Throughout this section, port \texttt{2A0} is used as the base I/O port address. If you configured the card with a different
port, you will need to adjust the command line arguments accordingly.

\subsection*{Initialization}

\texttt{I2C 2A0 i [clock] [addr]}

This command initializes the card at base I/O port \texttt{2A0}. If neither \texttt{[clock]} nor \texttt{[addr]} are provided,
the command will configure the card to assume a 4.43 MHz ISA bus clock, with a 90 kHz \itwoc{} \texttt{SCL} clock frequency.

Valid values for \texttt{[clock]} are 3, 4 (4.43 MHz), 6, 8, or 12.

Valid values for \texttt{[addr]} are \texttt{0} to \texttt{7F}. This address must be specified as a hexadecimal number.

\subsection*{Writing Data}

\texttt{I2C 2A0 w [addr] [data]}

This command writes data to the \itwoc{} device at address \texttt{[addr]}.

Data bytes are specified as hexadecimal numbers, separated by spaces. For example, \texttt{00 01 AB FE}.

\subsection*{Reading Data}

\texttt{I2C 2A0 r [addr] [count]}

This command reads \texttt{[count]} bytes from the \itwoc{} device at address \texttt{[addr]} and emits it to the console.

\subsection*{Status}

\texttt{I2C 2A0 s}

This command polls the \itwoc{} controller and emits status information to the console.

\section*{An Example}

The following example uses \texttt{I2C.EXE} to write digits to a 7-segment display via a HT16K33 controller.

\begin{verbatim}
@ECHO off

REM Initialize I2C Controller
I2C 2A0 i 4 AA

REM Initialize HT16K33 Controller
I2C 2A0 w 70 21
I2C 2A0 w 70 81
I2C 2A0 w 70 EF

REM Blank all segments
I2C 2A0 w 70 00 00 00 00 00 00 00 00 00 00 00 00 00 00 00 00 00

REM Display "1234"
I2C 2A0 w 70 00 06
I2C 2A0 w 70 02 5B
I2C 2A0 w 70 04 4F
I2C 2A0 w 70 06 66
\end{verbatim}

\end{document}
